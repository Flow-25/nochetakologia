\newcommand{\Wn}{$\mathscr{W}_n$}
\chapter{Mierzyć niezmierzone, czyli współczynnik nochetaczności}

\begin{quote}
\small
\itshape
Krwawym okiem z ukosa na rydwan spoziera, \\
Jakby chciał zająć puste miejsce Lucypera, \\
Warkocz długi w tył rzucił, i część nieba trzecią \\
Obwinął nim, gwiazd krocie zagarnął jak siecią, \\
I ciągnie je za sobą, a sam wyżéj głową \\
\textbf{Mierzy}, na północ, prosto w gwiazdę biegunową. \footfullcite{czlonek1834}
\end{quote}

Gatunek ludzki od swojego powstania ma wbudowaną ciekawość. Właśnie ona dała początek naukom wszelakim, dzięki czemu narodziła się matematyka. Odkąd człowiek zaczął operować na liczbach pojawiła się w nim potrzeba wyrażania wszystkiego właśnie przez nie. Nie inaczej było z nochetakami. Po ich spostrzeżeniu nastąpił naturalny acz podświadomy podział rozróżniający moc działania danego nochetaka\footfullcite{plemienne_nochetaki}. Wystąpiły wielokrotne próby wyrażenia ich mocy za pomocą wzorów, jednak udawały się one jedynie w bardzo wąskich zakresach\footnote{np. zawschodnich}, czy raczej zawężone do danego rodzaju nochataka. Właśnie ową "moc" we współczesnych czasach nazywamy współczynnikiem nochetaczności (ozn. $\mathscr{W}_n$). 
Czym dokładnie jest wspomniany skalar? Jest on podobny do współczynnika tarcia (również mnoży sie go najczęściej przez wektor siły) jednak ma dużo szersze zastosowania. Jest to współczynnik siły działającej przeciwnie do kierunku wektora siły danego obiektu, przy czym siłę rozumiemy nie tylko w sposób fizyczny ale także psychiczny, duchowy, parapsychiczny czy paradoksalny. Zwróćmy uwagę na narzucający się podział nochetaków na te o współczynniku większym od 1 (nochetaki prawdziwe) dodatnim ale mniejszym od 1 (nochetaki pozorne) i pozostałe (antynochetaki\footnote{W niektórych publikacjach (szczególnie angielskojęzycznych) możemy także natrafić na określenie \textit{Chetak}. My zostaniemy przy tradycyjnej wersji albowiem "TU JEST POLSKA!".}). 

\section{Współczynnik w historii}

Pierwsze wzmianki o porównaniach siły działania danego nochetaka odnajdujemy w pracach greckich filozofów takich jak Konetachos z Tesaloniki. W swojej pracy \footfullcite{konetachos} analizuje on szczegółowo wpływ własnej masy ciała na zachowanie swojej żony. Przełomowym odkryciem okazuje się, że im większa masa tym jego małżonka mniej przebywa w domu, pomimo usilnych prób powstrzymania jej ze strony autora. Widzimy ewidentnie negatywny wpływ masy ciała na czynnik \Wn. Historycznie rzecz biorąc, owy rękopis przyjmuje się być pierwszym spisanym i sklasyfikowanym badaniem zależności współczynnika \Wn. Kolejnymi badaniami porównującymi nochetaczność obiektów, w sposób bardziej naukowy, zajmowali się ludzie dużo później, bo dopiero w 500 r. 
\subsection{Badania pierwsze}
    Drugie i trzecie badanie należą do człowieka imieniem Serverus Clientus. Niewiele posiadamy informacji na jego temat. Wiemy jedynie że mieszkał na terenach obecnej Hiszpanii, a na śniadania jadał kaszankę \footfullcite{client_server}. Zachowały się rękopisy jego badań, w których to przedstawia ideę współczynnika nochetaczności i przeprowadza pierwsze w pełni poprawne badania dla wszystkich nochetaków z kategorii drzew\footfullcite{drzewa_nochetaki} jakie znajdowały się w okolicy dzisiejszego Madrytu. Proces badawczy przebiegał w następujący sposób. Rozpędziwszy się do swojej maksymalnej prędkości ($7\frac{m}{s}$) wbiegał w drzewo i po ocknięciu się sprawdzał w jakiej odległości od drzewa się znalazł. Eksperyment powtarzał 3-4 razy na danym gatunku drzewa. Zapisane wyniki przedstawił w swojej pracy, dokładnie opisując warunki eksperymentu. Przedstawiamy jedynie część wyników S. Clientusa, które uznaliśmy za najbardziej istotne:
\begin{table}[ht]
\centering
\caption{Wyniki badań S. Clientusa}
\label{tab:s_clientus}
\tiny % zmniejszenie czcionki
\begin{tabular}{|ll|ll|}
\hline
\textbf{Drzewo iglaste}        & \textbf{\Wn} & \textbf{Drzewo liściaste}     & \textbf{\Wn} \\
\hline
sosna zwyczajna                & 1.24           & dąb szypułkowy                & 1.98 \\
cedr libański                  & 1.71           & buk zwyczajny                 & 1.12 \\
sosna nadmorska                & 1.25           & klon zwyczajny                & 0.23 \\
świerk pospolity               & 1.18           & grab pospolity                & 0.91 \\
jodła kalifornijska            & 1.39           & jesion wyniosły               & 0.47 \\
jodła jeno syr                 & 1.39           & jesiotr bułgarski             & 0.47 \\
modrzew europejski             & 1.31           & topola biała                  & 0.95 \\
sosna alepska                  & 1.22           & dąb ostrolistny               & 1.14 \\
sosna czarna                   & 1.29           & kasztan jadalny               & 0.49 \\
sosna czarniejsza              & 1.52           & kasztan tydalny               & 0.56 \\
sosna murzyńska                & 1.52           & kasztan banalny               & 0.56 \\
jałowiec pospolity             & 1.08           & oliwka europejska             & 1.04 \\
cis pospolity                  & 1.17           & korkowiec amurski             & 1.02 \\
sosna pinii                    & 1.26           & platan klonolistny            & 1.11 \\
\hline
\end{tabular}
\end{table}

Fascynujące zdaje się ze drzewa iglaste mają z natury dużo większą nochetaczność niż drzewa liściaste, jednak to właśnie w tej drugiej grupie mamy lidera w postaci dębu szypułkowego. Sam S. Clientus komentuje ten wynik w następujący sposób
\begin{quote}
    \itshape
    %TODO 
    TODOD TODOD TODOD  TODOD TODOD TODOD TODOD  TODODTODOD TODOD TODOD  TODODTODOD TODOD TODOD  TODODTODOD TODOD TODOD  TODOD
\end{quote}

Badanie piąte, siódme i jedenaste zostały przeprowadzone w podobnym duchu. Na szczególną uwagę zasługuje właśnie badanie nr 7 analizujące nochetaczność wszelakiego rodzaju ksiąg, a także pieczęci na nich.  

\subsection{Badania złożone}

TODO TODO TODO Sfragistyka https://pl.wikipedia.org/wiki/Sfragistyka



Co ciekawe te badania miały zupełnie inną formułę, niźli badania pierwsze. Badanie 4 i 6 skupiały się głównie wokół symboli religijnych pośród plemion afrykańskich\footfullcite{znowu_afrykanie}.
Nie stanowią one tematu tej pracy, dlatego je pomijamy. Zaintrygowanych czytelników zachęcamy do popytania lokalnych księgarni i bibliotek o komentarz do tych badań\footnote{Z doświadczenia wiadomo, że niektórzy sprzedawcy patrzą się krzywo po spytaniu ich o posiadanie tego typu publikacji. Dzieje się tak albowiem nie są rozeznani w temacie. W takim przypadku zaleca się zatańczenie makareny i rytmicznym krokiem opuszczenie lokalu.}. Fascynujące zdaje się badanie 8 przeprowadzone przez grupę sfragistyków\footfullcite{sfragistycy}. Zwrócili oni uwagę na skład pieczęci woskowych, które to miały być dowodem poufności listów. Przeprowadzili skrupulatne badania, w których dowiedli, że najskuteczniejszym woskiem był wosk \textit{cerumen}. Drugi w kolejności był wosk pszczeli zabarwiony na żółto. Im żółtszy tym bardziej nochetaczny.  

\section{Charakterystyka danych i metodologia}


\begin{table}[ht]
\centering
\caption{Opis zmiennych użytych w modelu regresji}
\label{tab:opis_zmiennych}
\tiny % zmniejszenie czcionki
\begin{tabular}{p{4cm}p{2.5cm}p{1.5cm}}
\hline
\textbf{Zmienna} & \textbf{Opis} & \textbf{Jednostka} \\
\hline
Prędkość                        & ciągła                                & $\frac{m}{s^2}$\\
Czas oddziaływania              & ciągła                                & s\\
Temperatura                     & ciągła                                & $^\circ C$\\ 
Siła                            & ciągła                                & N \\
Masa                            & ciągła                                & kg\\
Moc                             & ciągła                                & $\mu$-organizmy\\
Kiełbasa                        & 0=krakowska, 1=śląska, 2=warszawska   & -\\
Stan skupienia                  & 0=stały, 1=ciekły, \mbox{2=uciekły}   & -\\
Stan rozproszenia               & 0=skupiony, \mbox{1=rozproszony}      & -\\
Współczynnik tarcia statycznego & ciągła                                & -\\
Liczba włosów                   & dyskretna                             & -\\
Liczba kończyn                  & dyskretna                             & -\\
Odległość od Słońca             & ciągła                                & km\\
Występowalność w dziełach Mickiewicza   & 0=brak, 1=pośrednia, 2=bezpośrednia   & -\\
Obiekt kultu                    & 0=nie jest, 1=jest                    & -\\
Całkowita energia obiektu       & ciągła                                & kcal\\
Długość liter w nazwie polskiej & dyskretna                             & -\\
Wytrzymałość                    & ciągła                                & s\\
Nieznośność                     & ciągła                                & dB\\
Nośność (np. ściany)            & ciągła                                & dB$^{-1}$\\
Radość z życia                  & ciągła                                & s\\
Liczba stron (zwykle dwie)      & dyskretna                             & -\\
Liczba publikacji o obiekcie    & dyskretna                             & -\\
Odcień żółci                    & ciągła                                & -\\
Kierunek świata                 & 0=północ, 1=wschód, 2=południe, \mbox{3=zachód}  & -\\
Opór aerodynamiczny             & ciągła                                & N\\
Opór elektryczny                & ciągła                                & $\Omega$\\
Opór daremny                    & ciągła                                & N\\
\hline
\end{tabular}
\end{table}




